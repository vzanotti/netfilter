L'objectif du projet était de développer un module \verb+netfilter+ pour réaliser du filtrage applicatif. Nous avions le choix entre deux angles d'attaques possibles : réaliser un module noyau pour \verb+netfilter+, ou bien réaliser un extension de \verb+netfilter+ en userspace, ce qui est d'ailleurs la stratégie employée par la future version de \verb+l7-filter+).

Nous avons choisi la solution userspace, que nous croyons clairement préférable ; elle est en particulier plus simple à développer (pas besoin de redémarrer le noyau pour chaque test échoué), elle est plus fiable (un crash du démon userspace ne fait pas paniquer le noyau), et elle permet enfin d'utiliser toutes les librairies existantes, en particulier les librairies d'expressions régulières standard (ce que ne permet pas un module noyau).\\

Compte tenu des similarités entre les deux projets, nous nous sommes inspirés de \verb+l7-filter-userspace+ \cite{RW}.
