L'objectif du projet, tel que défini par le sujet, était la réalisation d'une extension
pour \verb+netfilter+, afin de permettre un filtrage applicatif (en particulier un filtrage
pour les protocoles \verb+http+ et \verb+ftp+).

L'interêt du projet est de pouvoir différencier différents types de connexion http ou ftp,
pour permettre, entre autres, de les interdire, ou de limiter leur débit. Par exemple, l'extension
devrait permettre de repérer « les requêtes de type \verb+GET+ sur des fichiers PDF en http ».

L'extension devrait donc permettre une utilisation du type :
\begin{itemize}
\item \verb+iptables -A FORWARD -p tcp -m webfilter --function get+\\ \verb+--urlsizemax 255 -j ACCEPT+ : le module \og webfilter \fg{} n'autorise que la fonction \verb+GET+ avec une URL de taille max 255 ;
\item \verb+iptables -A FORWARD -p tcp -m ftpfilter --function get+\\
\verb+--filetype pdf -j ACCEPT+\\
\verb+iptables -A FORWARD -p tcp -m ftpfilter --function list+\\
\verb+--filetype pdf -j ACCEPT+ : le module \og webfilter \fg{} n'autorise que les fonctions \verb+LIST+ et \verb+GET+ sur les fichiers PDF.
\end{itemize}

La possibilité de journaliser les évènements est aussi primordiale. En effet, il est indispensable que des évènements puissent être remontés en fonction des critères de filtrage. La cause du rejet d'une requête doit donc être spécifiée, par exemple, \og URL de taille $>$ 255 \fg{}.
