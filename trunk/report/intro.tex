Pour ce projet, il y avait deux angles d'attaques possibles : réaliser un module noyau pour netfilter, ou bien réaliser un extension de netfilter en userspace (c'est la stratégie employée par la future version de l7-filter).
Nous avons choisi la solution userspace car nous pensons qu'elle est clairement préférable ; elle est en particulier plus simple à développer (pas besoin de
redémarrer le noyau pour chaque test échoué), elle est plus fiable (un
crash du démon userspace ne fait pas paniquer le noyau), et elle
permet d'utiliser toutes les librairies existantes, en particulier les
librairies de regexp (ce que ne permet pour ainsi dire pas un module
noyau).\\

Pour notre projet nous nous sommes inspiré de l7-filter-userspace \cite{RW}.