\section<presentation>[Principes et risques]{Les principes et les risques du filtrage}
  \begin{frame}
    \frametitle{Contexte: un filtre transparent}
    \begin{itemize}
      \item<1> Filtrer du traffic (web) sans toucher aux connexions TCP.
        % image filter-before.png
      \item<2> Filtrer du traffic (web) sans toucher aux connexions TCP.
        % image filter-after.png
      \item<3-> Cas d'�cole: appliquer des r�gles de QoS restrictives sur les fichiers {\tt .pdf}.
      \item<4-> Hypoth�se: l'un des cot� est neutre (pas de complicit�).
    \end{itemize}
  \end{frame}

  \begin{frame}
    \frametitle{Fonctionnement d'un filtre applicatif}
    Fonctionnement d'un filtre transparent au niveau 7 (OSI):
    \begin{itemize}
      \item Identification et groupage des packets par session TCP.\\
          {\tt\small [tcp src=129.104.201.51 dst=87.98.164.72 sport=4242 dport=80]} ou \\
          {\tt\small [tcp src=87.98.164.72 dst=129.104.201.51 sport=80 dport=4242]}
      \item Extraction du contenu des buffers egress/ingress.\\
        {\tt\small > GET /abcd.pdf HTTP/1.0}\\
        {\tt\small > Host: www.blih.org}\\
        {\tt\small > }\\
        {\tt\small < HTTP/1.1 200 OK}\\
        ...
      \item D�tection du protocole utilis�.
      \item Extraction des information (protocol buffer).\\
        {\tt\small proto=HTTP, method=GET, url=/abcd.pdf}
      \item Classification finale.\\
        Exemple de r�gle: {\tt\small proto=HTTP method=GET url=.*\.pdf}
    \end{itemize}
  \end{frame}

  \begin{frame}
    \frametitle{Attaques et contournement d'un filtre applicatif}
    \begin{itemize}
      \item Contournement par fragmentation des paquets
      \item Contournement par r�-arrangement des paquets TCP
      \item Contournement par exploitation du protocole
      \item Contournement par leurre du d�tecteur de protocole
      \item Contournement par non-respect du protocole
      \item Attaque \emph{Denial of Service} classique
    \end{itemize}
  \end{frame}

  \begin{frame}
    \frametitle{Contournement par fragmentation des paquets}
    Id�e: utiliser le protocole IP pour casser ``l'atomicit�'' des �changes.
    % FRAG flag, ou paquets � payload r�duit.
  \end{frame}

  \begin{frame}
    \frametitle{Contournement par r�-arrangement des paquets TCP}
    Id�e: changer l'ordre des paquets TCP pour obfusquer le buffer \emph{egress}.
  \end{frame}

  \begin{frame}
    \frametitle{Contournement par exploitation du protocole}
    Id�e: exploiter des fonctionnalit�s obscures (ou pas) du protocole.
    % Keepalive
  \end{frame}

  \begin{frame}
    \frametitle{Contournement par leurre du d�tecteur de protocole}
    Id�e: comettre des erreurs r�cup�rables de nature � leurrer le d�tecteur.
    % Commencer par une erreur en SMTP pour �viter le EHLO.
    % Utiliser des lignes de requete extremement longues, type /././././.....
  \end{frame}

  \begin{frame}
    \frametitle{Contournement par non-respect du protocole}
    Id�e: utiliser des \emph{flavor} peu-connues ou des tol�rances du protocole.
    % Par exemple utiliser "GET <url>\n"
  \end{frame}

  \begin{frame}
    \frametitle{Attaque \emph{Denial of Service} classique}
    Id�e: exploiter les impl�mentations na�ves du protocole.
  \end{frame}
